\chapter{Automatização de processos}\label{chp:conceitos_basicos}


\section{Processos}\label{sec:conceitos_basicos-processos}
Processo de negócio é um conjunto de atividades coordenadas, relacionadas entre si, que envolvem diferentes pessoas, procedimentos, áreas e tecnologias com o objetivo de gerar valor para a empresa, seja em forma de produtos ou serviços, internos ou externos.

\section{BPM}\label{sec:conceitos_basicos-bpm}
BPM é o acrônimo para o inglês Business Process Management, ou gestão de processos de negócio em português. Seu principal objetivo é oferecer uma abordagem sistemática para a execução, adaptação e melhoria de processos de negócio em um ambiente de constantes mudanças. O BPM pode ser encarado sob duas perspectivas distintas: o BPM como engenharia de software ou o BPM como disciplina de gestão.

\section{BPMn}\label{sec:automatizacao-processos-bpmn}

\section{BPMS}\label{sec:automatizacao-processos-bpms}
\subsection{Activiti BPM}\label{sec:automatizacao-processos-bpms-activiti}

O Activiti BPM é uma ferramenta BPMS open-source, utilizada na automatização de processos de negócio em um sistema de informação. Tem por objetivo prover um motor BPM leve, estável e fácil de integrar em diferentes tipos de aplicações e ambientes. A versão 5.19 foi utilizada neste trabalho.



\section{Redmine}\label{sec:conceitos_basicos-redmine}
O Redmine\cite{redmine} é uma ferramenta de gerenciamento de projetos open-source. Foi criada por Jean-Philippe Lang em 2006. Desenvolvido em Ruby, utilizando a framework Rails, tem como objetivo dar flexibilidade de configuração ao usuário, e também ao desenvolvedor. A versão 3.1 deste software foi utilizada neste trabalho.


