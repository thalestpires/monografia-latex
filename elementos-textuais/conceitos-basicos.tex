\chapter{Conceitos básicos}\label{chp:conceitos_basicos}

\section{Introdução}\label{sec:conceitos_basicos-introducao}
Neste capítulo introduzimos alguns conceitos básicos que serão abordados neste trabalho, contextualizando o leitor no tema, de modo que qualquer um seja capaz de absorver o conteúdo do texto e apreciar a solução proposta, bem como as conclusões apresentadas.

\section{Processos}\label{sec:conceitos_basicos-processos}
Processo de negócio é um conjunto de atividades coordenadas, relacionadas entre si, que envolvem diferentes pessoas, procedimentos, áreas e tecnologias com o objetivo de gerar valor para a empresa, seja em forma de produtos ou serviços, internos ou externos.

\section{BPM}\label{sec:conceitos_basicos-bpm}
BPM é o acrônimo para o inglês Business Process Management, ou gestão de processos de negócio em português. Seu principal objetivo é oferecer uma abordagem sistemática para a execução, adaptação e melhoria de processos de negócio em um ambiente de constantes mudanças. O BPM pode ser encarado sob duas perspectivas distintas: o BPM como engenharia de software ou o BPM como disciplina de gestão.

\section{Activiti BPM}\label{sec:conceitos_basicos-activiti}


\section{Redmine}\label{sec:conceitos_basicos-redmine}
O Redmine é uma ferramenta de gerenciamento de projetos open-source. Foi criada por Jean-Philippe Lang em 2006. Desenvolvido em Ruby, utilizando a framework Rails, tem como objetivo dar flexibilidade de configuração ao usuário, e também ao desenvolvedor. A versão 3.1 deste software foi utilizada neste trabalho.


