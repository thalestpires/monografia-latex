\chapter{Integração Redmine e Activiti BPM}\label{chp:integracao_redmine_activiti}

\section{Introdução}\label{sec:integracao_redmine_activiti-introducao}
Dadas as vantagens e limitações das ferramentas Activiti e Redmine utilizadas como plataformas para a automatização de processos, decidimos desenvolver uma forma de integrá-las para explorarmos os pontos positivos de cada uma delas.

Para atingir este objetivo, criamos um plugin para o Redmine que possibilita a comunicação com o Activiti. Nas próximas sessões, descreveremos o processo de construção deste plugin e como utilizá-lo.


\section{Implementação}\label{sec:integracao_redmine_activiti-implementacao}

A integração tem por objetivo centralizar o máximo de funcionalidades no Redmine, deixando transparente tanto para o usuário comum como o gestor, a existência de um motor BPM por trás.

\subsection{Funcionalidades}\label{sec:integracao_redmine_activiti_implementacao_funcionalidades}

\subsubsection{Conexão com o Activiti }\label{sec:integracao_redmine_activiti_inplementacao_funcionalidades_conexão}
Para iniciar a utilização do plugin é necessário configurar os detalhes para conexão com o Activiti BPM.
Após colocar o servidor, login e senha, também é preciso disparar os jobs que rodam continuamente para sincronizar os processos e tarefas entre o Redmine e o Activiti.

\subsubsection{Configuração dos estados}\label{sec:integracao_redmine_activiti_inplementacao_funcionalidades_conexão}
Também é necessário configurar os estados principais a serem utilizados pelo plugin.
Configure os estados padrão que devem ser utilizados quando um processo é criado, concluído ou está em andamento.

\subsubsection{Deploy de um processo}\label{sec:integracao_redmine_activiti_inplementacao_funcionalidades_deploy}
Como já foi explicado, o processo deve ser modelado na notação BPMN. Mas à partir deste ponto, toda a interação será através do Redmine.
Na tela de processos, ao clicar em Novo processo, é possível fazer upload de uma modelagem de um processo para o Redmine. Esta operação tenta o deploy do arquivo selecionado no Activiti, através do serviço REST e apresenta o resultado na tela.

A cada vez que esta tela é atualizada, a lista de processos é atualizada. Nela são mostrados todo os fluxos de trabalho ativos no BPMS, inclusive os que não foram instalados através da interface citada acima. Ao clicar numa linha, o usuário consegue visualizar o diagrama do processo.

\subsubsection{Configuração de um processo}\label{sec:integracao_redmine_activiti_inplementacao_funcionalidades_configuracao}
A integração das duas ferramentas se dá na sincronização dos processos e tarefas humanas do Activiti.

\subsubsubsection{Representação de um processo}\label{sec:integracao_redmine_activiti_inplementacao_funcionalidades_configuracao_processo}
ddd

\section{Resultados}\label{sec:integracao_redmine_activiti-resultados}