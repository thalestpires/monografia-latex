\chapter{Problema}\label{chp:problema}

\section{Introdução}\label{sec:problema-introducao}

As organizações, sejam de grandes dimensões ou não, são constituídas por recursos humanos e não-humanos que interagem entre si, através do compartilhamento de recursos físicos ou virtuais, como por exemplo na troca de informações operacionais entre diferentes áreas de uma empresa. Cada uma dessas interações definem processos de negócio que tem por objetivo produzir um resultado para os envolvidos nessa interação extra ou intra-organizacional.

Os processos de negócio, quando realizados de forma desorganizada e despadronizada, levam à ineficiência organizacional pelo simples fato de sua execução não otimizada. Isso ocasiona em elevação dos custos, aumento no retrabalho, insatisfação dos colaboradores e a consequente diminuição na qualidade dos serviços relacionados. Portanto, torna-se altamente necessário uma boa gestão dos processos, a fim de que esses problemas sejam evitados ou mitigados a tempo e não tornem-se um câncer corporativo.

O acesso mais rápido e eficiente às informações estabelece melhores condições para a execução de processos de negócio. Entretanto, não é suficiente para evitar problemas inerentes à presença de informação. A execução despadronizada de processos sem o apoio de um sistema de informação, ou mesmo suportada por sistemas de informação que não acompanham a flexibilidade da constante mudança dos processos, são em geral gargalos fundamentais no insucesso das diversas organizações existentes. A solução desses problemas estabelece melhores condições para a execução, monitoramento e controle de processos, o que termina por aperfeiçoar a tomada de decisão no nível estratégico organizacional.

Neste trabalho, apresentaremos uma alternativa para garantir que a execução das atividades das empresas sejam realizadas da maneira esperada e aumentar a capacidade de monitoramento de cada uma das etapas, que é a automatização de processos. 

\section{Automatização de processos}\label{sec:problema-automatizacao_processos}



