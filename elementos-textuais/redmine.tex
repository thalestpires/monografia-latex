\chapter{Redmine}\label{chp:redmine}

\section{Introdução}\label{sec:redmine-introducao}
Neste capítulo vamos explicar como o Redmine, uma ferramenta de gerenciamento de projetos foi utilizada para gestão de processos, ilustrando com um exemplo. Vamos apresentar ainda, as capacidade extensiva desta ferramenta, através do desenvolvimento de plugins. Por último vamos abordar as limitações do Redmine, que nos motivaram a desenvolver algo novo para atingir nosso objetivo em automatização de processos.

\section{Gestão de processos com o Redmine}\label{sec:redmine-gestao_processos}


\section{Como automatizar um processo?}\label{sec:redmine-automatizar_processo}

\section{Plugins}\label{sec:redmine-plugins}
O Redmine foi desenvolvido de forma a ser extensível por meio de plugins. É possível modificar um funcionalidade da ferramenta, ou criar novas funcionalidades sem precisar alterar o código desta. Os plugins são desenvolvidos em Rails, a mesma linguagem de programação do Redmine. 

Para possibilitar extensões de funcionalidades que envolvem enxertar pedaços de código no meio de uma classe ou de uma tela, o Redmine disponibiliza hooks em diversas partes da ferramenta. São tags com um identificador da parte do código em que estão inseridas. E para utilizar este hook basta incluir um hook listener num plugin, e direcionar qual arquivo ou método um determinado hook vai disparar.

\section{Limitações}\label{sec:redmine-limitacoes}

