\chapter{Redmine}\label{chp:redmine}

\section{Introdução}\label{sec:redmine-introducao}
Neste capítulo vamos explicar como o Redmine, uma ferramenta de gerenciamento de projetos foi utilizada para gestão de processos, ilustrando com um exemplo. Vamos apresentar ainda, a capacidade extensiva desta ferramenta através do desenvolvimento de plugins. Por último vamos abordar as limitações do Redmine que nos motivaram a desenvolver algo novo para atingir nosso objetivo em automatização de processos.

\section{Estrutura básica do Redmine}\label{sec:redmine-estrutura_basica}

A estrutura básica de gerenciamento de projetos no Redmine é composta por X principais elementos. São eles:

\subsection{Tarefas}\label{subsection:redmine-estrutura_basica-tarefa}

São as unidades básicas de execução de trabalho dos projetos (\ref{subsection:redmine-estrutura_basica-projeto}). Elas contém os dados relevantes para o seu gerenciamento (e.g, tipo (\ref{subsection:redmine-estrutura_basica-tracker}, situação(\ref{subsection:redmine-estrutura_basica-status}), data de início, data de conclusão, categoria), para a sua execução (e.g, título, descrição) e dados adicionais que podem variar entre os projetos e tipos de tarefa, que são cadastrados como campos personalizados (\ref{subsection:redmine-estrutura_basica-custom_fields}).  

\subsection{Projetos}\label{subsection:redmine-estrutura_basica-projeto}

\subsection{Tipos de tarefa}\label{subsection:redmine-estrutura_basica-tracker}

\subsection{Situação}\label{subsection:redmine-estrutura_basica-status}

\subsection{Papéis de usuários}\label{subsection:redmine-estrutura_basica-role}

\subsection{Campos Customizados}\label{subsection:redmine-estrutura_basica-custom_fields}


\section{Gestão de processos com o Redmine}\label{sec:redmine-gestao_processos}


\section{Como automatizar um processo?}\label{sec:redmine-automatizar_processo}

\section{Plugins}\label{sec:redmine-plugins}
O Redmine foi desenvolvido de forma a ser extensível por meio de plugins. É possível modificar um funcionalidade da ferramenta, ou criar novas funcionalidades sem precisar alterar o código desta. Os plugins são desenvolvidos em Rails, a mesma linguagem de programação do Redmine. 

Para possibilitar extensões de funcionalidades que envolvem enxertar pedaços de código no meio de uma classe ou de uma tela, o Redmine disponibiliza hooks em diversas partes da ferramenta. São tags com um identificador da parte do código em que estão inseridas. E para utilizar este hook basta incluir um hook listener num plugin, e direcionar qual arquivo ou método um determinado hook vai disparar.

\section{Limitações}\label{sec:redmine-limitacoes}

