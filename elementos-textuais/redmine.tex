\chapter{Redmine}\label{chp:redmine}

\section{Introdução}\label{sec:redmine-introducao}
Neste capítulo vamos explicar como o Redmine, uma ferramenta de gerenciamento de projetos foi utilizada para gestão de processos, ilustrando com um exemplo. Vamos apresentar ainda, a capacidade extensiva desta ferramenta através do desenvolvimento de plugins. Por último vamos abordar as limitações do Redmine que nos motivaram a desenvolver algo novo para atingir nosso objetivo em automatização de processos.

\section{Estrutura básica do Redmine}\label{sec:redmine-estrutura_basica}

A estrutura básica de gerenciamento de projetos no Redmine é composta por 6 principais elementos. São eles:

\subsection{Tarefas}\label{subsection:redmine-estrutura_basica-tarefa}

São as unidades básicas de execução de trabalho dos projetos (\ref{subsection:redmine-estrutura_basica-projeto}). Elas contém os dados relevantes para o seu gerenciamento (e.g, tipo (\ref{subsection:redmine-estrutura_basica-tracker}), situação (\ref{subsection:redmine-estrutura_basica-status})), para a sua execução (e.g, título, descrição) e dados adicionais que podem variar entre os projetos e tipos de tarefa, que são cadastrados como campos personalizados (\ref{subsection:redmine-estrutura_basica-custom_fields}).  

\subsection{Projetos}\label{subsection:redmine-estrutura_basica-projeto}

Projetos são o objeto central do Redmine. Eles são compostos por diversos módulos que acrescentam diferentes dimensões para o seu gerenciamento, como gerenciamento de tarefas, planejamento de versões, acompanhamento do progresso das tarefas em um diagrama de Gantt, Wiki para organização do conhecimento, entre outros. 

\subsection{Tipos de tarefa}\label{subsection:redmine-estrutura_basica-tracker}

Tipos de tarefa definem o fluxo de trabalho para a realização de atividades similares. De acordo com o tipo, variam as informações necessárias para a execução da tarefa, a sequência de passos para sua conclusão e as ações que cada membro do projeto pode desempenhar em cada etapa.

\subsection{Situação}\label{subsection:redmine-estrutura_basica-status}

Situação indica em qual etapa do processo uma tarefa se encontra. É usada para definir as ações possíveis pelos membros do projeto. Essas ações podem ser diferentes para cada tipo de tarefa.

\subsection{Papéis de usuários}\label{subsection:redmine-estrutura_basica-role}

Os papéis de usuários definem quais permissões um usuário possui, como, por exemplo, visualizar, adicionar e editar tarefas, editar Wiki, adicionar e editar documentos. Os papéis são atribuídos aos usuários em cada projeto que ele participa, portanto, ele pode possuir permissões diferentes dependendo do projeto de que ele é membro.

\subsection{Campos Personalizados}\label{subsection:redmine-estrutura_basica-custom_fields}

Todas as tarefas de projetos possuem dados em comum, como data de início, data de término e descrição. No entanto, dependendo das especificidades de um projeto ou tipo de tarefa, podem ser necessárias informações adicionais. Para esses casos existem os campos personalizados. Eles permitem criar novos campos e adicioná-los às tarefas, estendendo as configurações padrão da ferramenta.

\section{Gestão de processos com o Redmine}\label{sec:redmine-gestao_processos}

A estrutura de projetos do Redmine é altamente configurável. Todos os elementos explicados na seção \ref{sec:redmine-estrutura_basica} são cadastrados pelos usuários administradores do sistema, que são responsáveis por configurá-los e personalizá-los para atender às demandas de cada projeto.

Todas as configurações e personalizações citadas no último parágrafo são feitas exclusivamente pela interface da ferramenta, sem necessidade de alterações no código da aplicação ou arquivos de configuração, o que confere aos administradores capacidade para modelar a estrutura da ferramenta da forma que for mais conveniente para a necessidade dos usuários.

Devido a sua natureza adaptável e ferramentas de gerenciamento de tarefas e de conhecimento, o Redmine pode ser utilizado, também, em outros contextos, como no gerenciamento de processos de negócio.


\section{Como automatizar um processo?}\label{sec:redmine-automatizar_processo}

\section{Plugins}\label{sec:redmine-plugins}
O Redmine foi desenvolvido de forma a ser extensível por meio de plugins. É possível modificar um funcionalidade da ferramenta, ou criar novas funcionalidades sem precisar alterar o código desta. Os plugins são desenvolvidos em Rails, a mesma linguagem de programação do Redmine. 

Para possibilitar extensões de funcionalidades que envolvem enxertar pedaços de código no meio de uma classe ou de uma tela, o Redmine disponibiliza hooks em diversas partes da ferramenta. São tags com um identificador da parte do código em que estão inseridas. E para utilizar este hook basta incluir um hook listener num plugin, e direcionar qual arquivo ou método um determinado hook vai disparar.

\section{Limitações}\label{sec:redmine-limitacoes}

