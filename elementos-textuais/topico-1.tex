\chapter{Introdução}\label{chp:LABEL_CHP_1}

\section{Motivação}\label{sec:LABEL_CHP_1_SEC_A}
Os processos de negócio consistem numa sequência de atividades e serviços que  encadeados cumprem determinado objetivo ou função na organização em que é desempenhado. A automatização de processos de negócio consiste na aplicação de tecnologia, de forma que uma ou mais atividades de um processo possam ser automatizadas, reduzindo assim a dependência de atuação humana para sua execução.\footnote{\url{http://www.lipsum.com/}}.

\section{Objetivos}\label{sec:LABEL_CHP_1_SEC_B}
O objetivo geral deste trabalho é propor uma solução para automatização de processos complexos que também conta com interação humana, inclusive durante o fluxo destes. Vamos apresentar uma ferramenta que permite a modelagem de fluxos variados, facilita a configuração e implementação, bem como a utilização contínua por usuários com conhecimentos básicos de processo e o acompanhamento deste por um gerente.

\section{Organização do texto}\label{sec:LABEL_CHP_1_SEC_C}

\chapter{Conceitos básicos}\label{chp:LABEL_CHP_2}

\section{Introdução}\label{sec:LABEL_CHP_2_SEC_A}

\section{Processos}\label{sec:LABEL_CHP_2_SEC_B}
LUCAS VAI FAZER

\section{BPM}\label{sec:LABEL_CHP_2_SEC_C}
BPM é o acrônimo para o inglês Business Process Management, ou gestão de processos de negócio em português. Seu principal objetivo é oferecer uma abordagem sistemática para a execução, adaptação e melhoria de processos de negócio em um ambiente de constantes mudanças. O BPM pode ser encarado sob duas perspectivas distintas: o BPM como engenharia de software ou o BPM como disciplina de gestão.

\section{Activiti BPM}\label{sec:LABEL_CHP_2_SEC_D}


\section{Redmine}\label{sec:LABEL_CHP_2_SEC_E}
O Redmine é uma ferramenta de gerenciamento de projetos open-source. Foi criada por Jean-Philippe Lang em 2006. Desenvolvido em Ruby, utilizando a framework Rails, tem como objetivo dar flexibilidade de configuração ao usuário, e também ao desenvolvedor. A versão 3.1 deste software foi utilizada neste trabalho.


\chapter{Problema}\label{chp:LABEL_CHP_3}

\section{Introdução}\label{sec:LABEL_CHP_3_SEC_A}
LUCAS VAI FAZER

\section{Automatização de processos}\label{sec:LABEL_CHP_3_SEC_B}


\chapter{Redmine}\label{chp:LABEL_CHP_3}

\section{Introdução}\label{sec:LABEL_CHP_3_SEC_A}
Neste capítulo vamos explicar como o Redmine, uma ferramenta de gerenciamento de projetos foi utilizada para gestão de processos, ilustrando com um exemplo. Vamos apresentar ainda, as capacidade extensiva desta ferramenta, através do desenvolvimento de plugins. Por último vamos abordar as limitações do Redmine, que nos motivaram a desenvolver algo novo para atingir nosso objetivo em automatização de processos.

\section{Gestão de processos com o Redmine}\label{sec:LABEL_CHP_3_SEC_B}


\section{Como automatizar um processo?}\label{sec:LABEL_CHP_3_SEC_C}

\section{Plugins}\label{sec:LABEL_CHP_3_SEC_D}
O Redmine foi desenvolvido de forma a ser extensível por meio de plugins. É possível modificar um funcionalidade da ferramenta, ou criar novas funcionalidades sem precisar alterar o código desta. Os plugins são desenvolvidos em Rails, a mesma linguagem de programação do Redmine. 

Para possibilitar extensões de funcionalidades que envolvem enxertar pedaços de código no meio de uma classe ou de uma tela, o Redmine disponibiliza hooks em diversas partes da ferramenta. São tags com um identificador da parte do código em que estão inseridas. E para utilizar este hook basta incluir um hook listener num plugin, e direcionar qual arquivo ou método um determinado hook vai disparar.

\section{Limitações}\label{sec:LABEL_CHP_3_SEC_E}


\chapter{Activiti BPM}\label{chp:LABEL_CHP_4}

\section{Introdução}\label{sec:LABEL_CHP_4_SEC_A}
Criado em 2010 por ex-integrantes do projeto jBPM, o Activiti BPM é um projeto de código aberto sob a licença Apache 2, que provê um motor BPM leve e completo sob a especificação BPMN 2.0. O Activiti é desenvolvido sob a linguagem de programação Java e é facilmente integrável com aplicações existentes por sua leveza e API amigável.

\section{Vantagens}\label{sec:LABEL_CHP_4_SEC_B}

\section{Como automatizar um processo?}\label{sec:LABEL_CHP_4_SEC_C}

\section{Limitações}\label{sec:LABEL_CHP_4_SEC_D}


\chapter{Integração Redmine e Activiti BPM}\label{chp:LABEL_CHP_5}

\section{Introdução}\label{sec:LABEL_CHP_5_SEC_A}
LUCAS VAI FAZER

\section{Implementação}\label{sec:LABEL_CHP_5_SEC_A}

\section{Resultados}\label{sec:LABEL_CHP_5_SEC_A}